Depuis que j'ai commencé à enseigner en 2010, au lycée Jules Haag de Besançon, je me dis qu'un jour, il faudra que je partage mes cours, mes TP, mes TD pour remercier tous les gens grâce auxquels j'ai pu commencer à enseigner. 

Mais chaque année, c'est pareil : mes cours ne sont pas assez aboutis, et il y a sûrement des erreurs. Je n'ai pas rédigé le corrigé de toutes mes activités. Et puis d'ailleurs, toutes ces activités ne sont pas les miennes et sont glanées, sur des sites de collègues, sur le site de l'UPSTI etc. Et puis c'est le bazar dans mes ressources. 

Et puis je me dis aussi que je pourrais les partager sur un site web, alors j'en commence un, puis un second, puis un troisième... et au final je ne trouve jamais le temps de les alimenter en temps réel. 

Et puis, et puis, et puis, cela n'aboutit jamais. 

Donc cette année, c'est tant pis... mes ressources sont un peu plus organisées. Elles ne sont pas parfaites pour autant, ni terminées. Et puis je ne suis pas l'auteur de chacune d'entre elles... Tant-pis, je me lance, voila une partie des mes ressources.

Malheureusement tous ces documents ne sont pas toujours sourcés. J'espère que les auteurs initiaux ne m'en voudront pas et qu'ils seront au moins listés ci-dessous. Toutes les ressources n'ont pas forcément été testées devant élèves...  

Enfin, ce document n'ai pas vraiment vocation à être utilisé en tant que tel, ce qui se traduire aussi par quelques petits couacs de mise en page.

Je peux maintenant dresser une liste de remerciement sans fin...

Merci, donc aux collègues de mon premier lycée, l'équipe PTSI-PT du lycée Jules Haag de Besançon à savoir Maryline Carrez et André Grux. 

Merci aux collègues de mon second lycée, le Lycée Rouière de Toulon, Jean-Pierre Pupier, Patrick Beynet, Laurent Deschamps, François Jeay, Gilles Himmelspach, Marc Guillaume.

Merci aux collègues du lycée La Martinière Monplaisir, Viviane Reydellet, Patricia Bessonnat, Isabelle Cotta, Cédric Gamelon, Philppe Dubois, Antoine Martin, Emilien Durif.

Merci aux collègues qui partagent leur ressources en ligne : Florestan Mathurin, Stéphane Genouël et l'ensemble des collègues du Pôle Chateaubriand -- Joliot Curie, Partrick Dupas, merci !

Merci aux collègues qui partagent leurs ressources sur le site de l'UPSTI.

Merci à tous les collègues que j'ai pu (re)rencontrer au gré de mes diverses activités, Ivan Liebgott, Frédéric Mazet, Vincent Honorat, Hervé Guillermin, Robert Papanicola, Florian Bernardeau, Christophe Durant, François Weiss, Philippe Hautcoeur, Marc Libourel, Sébastien Gergadier, Franck Achard, Jean-Philippe Costes, Emmanuel Bigeard, David Violeau, Alain Caignot, David Fournier, Marc Derumaux, Damien Aza-Vallina, Christophe Grèze, Damien Guigues, Thomas Raulin, Thibaut Kovaltchouk, Augustin Hugues, Sébastien Grange... j'en oublie, j'espère qu'ils ne se fâcheront pas. Merci à ceux qui nous mettent des coups de pieds aux fesses et qui permettent de se questionner, d'avancer. 

Merci à ceux qui étaient amis avant d'être collègues, merci Renaud ;) (Costadoat) ou à ceux qui ont été collègues puis amis, salut Anthony ;) (Meurdefroid), resalut Pat et Fred !

Pas merci à Robert et à Anthony car à cause d'eux, je me dis qu'il faut que je refasse tous mes diagrammes de Bode et tous mes schémas cinématiques :).


Nos Sciences Industrielles de l'Ingénieur sont belles, on y rencontre plein de problèmes, plein de solutions,  (parfois des non solutions\footnote{J'ai toujours un problème avec les deux ponctuelles équivalentes à une liaison sphère-cylindre... mais où est l'axe ?}). 
On y rencontre surtout plein de profs qui aiment leur élèves, qui aiment leur métier, qui aiment leur discipline et qui aiment partager, expliquer, se questionner. Tous ces échanges m'ont beaucoup appris, me questionnent et m'apprennent encore beaucoup. 

Merci aux élèves qui nous rajeunissent, qui nous questionnent, qui nous pressent, qui nous énervent, pour qui on donne tant de temps, mais qui ne le voient pas :)

