\begin{enumerate}

    \item Créez un programme qui prend une liste de mots et détermine, pour chaque mot, si celui-ci est un palindrome. Le programme doit afficher les mots qui sont des palindromes.

    \item Écrivez un programme qui calcule la somme de tous les nombres premiers inférieurs à un nombre donné par l'utilisateur. Utilisez des boucles \texttt{for} imbriquées pour vérifier si chaque nombre est premier.

    \item Créez un programme qui prend deux chaînes de caractères et trouve les lettres communes entre les deux chaînes, sans répétition. Utilisez une boucle \texttt{for} pour comparer les lettres et construisez le résultat dans une liste ou une chaîne.

    \item Écrivez un programme qui génère la suite de Collatz pour un nombre donné. La suite de Collatz est définie par : si le nombre est pair, le diviser par 2 ; s'il est impair, le multiplier par 3 et ajouter 1. La suite continue jusqu'à ce que le nombre atteigne 1.

    \item Créez un programme qui prend une liste de listes représentant une matrice carrée et calcule la somme des éléments au-dessus de la diagonale principale.

    \item Écrivez un programme qui génère les premiers \texttt{n} nombres de la suite de Lucas, une variante de la suite de Fibonacci avec les termes initiaux \texttt{2} et \texttt{1}.

    \item Créez un programme qui prend une phrase et affiche chaque mot avec ses lettres en ordre alphabétique. Par exemple, pour "bonjour tout le monde", le programme devrait retourner "bjnoru ottu el ddmeno".

    \item Écrivez un programme qui trouve tous les nombres parfaits inférieurs à un nombre donné. Un nombre parfait est un nombre égal à la somme de ses diviseurs propres (comme 6, qui est 1 + 2 + 3).

    \item Créez un programme qui prend une liste de nombres et calcule le produit de tous les éléments pairs, tout en additionnant les éléments impairs. Affichez le produit des pairs et la somme des impairs.

    \item Écrivez un programme qui génère les \texttt{n} premières lignes du triangle de Pascal et les affiche sous forme de liste de listes. Utilisez une boucle \texttt{for} pour construire chaque ligne.

\end{enumerate}

