\begin{enumerate}

    \item \textbf{Calcul de la somme des entiers} \\
    Écrire un programme qui demande à l’utilisateur un nombre entier positif $n$ et qui utilise une boucle \texttt{while} pour calculer la somme des entiers de $1$ à $n$ inclus.
    
    \item \textbf{Trouver le plus petit multiple} \\
    Demandez à l’utilisateur de saisir deux nombres entiers positifs, $a$ et $b$. Utilisez une boucle \texttt{while} pour trouver le plus petit multiple commun de $a$ et $b$.
    
    \item \textbf{Calcul de la factorielle} \\
    Écrire un programme qui demande à l’utilisateur un nombre entier positif $n$ et utilise une boucle \texttt{while} pour calculer la factorielle de $n$, notée $n!$. Rappel : $n! = 1 \times 2 \times 3 \times \dots \times n$.
    
    \item \textbf{Inversion d'un nombre} \\
    Écrire un programme qui demande un nombre entier positif à l'utilisateur et utilise une boucle \texttt{while} pour inverser ses chiffres. Par exemple, si l'utilisateur entre $1234$, le programme doit afficher $4321$.
    
    \item \textbf{Deviner un nombre} \\
    Écrire un programme qui génère un nombre aléatoire entre $1$ et $100$. Le programme doit permettre à l’utilisateur de deviner le nombre en entrant des propositions successives. Utilisez une boucle \texttt{while} pour permettre à l’utilisateur de deviner jusqu'à ce qu'il trouve le bon nombre.
    
    \item \textbf{Nombre de chiffres} \\
    Écrire un programme qui demande à l’utilisateur un nombre entier positif et utilise une boucle \texttt{while} pour déterminer le nombre de chiffres dans ce nombre. Par exemple, si l'utilisateur entre $1234$, le programme doit afficher $4$.
    
    \item \textbf{Suite de Fibonacci} \\
    Écrire un programme qui affiche les $n$ premiers termes de la suite de Fibonacci, où $n$ est un nombre entier positif donné par l’utilisateur. Utilisez une boucle \texttt{while} pour calculer les termes. Rappel : La suite de Fibonacci commence par $0$ et $1$, et chaque terme suivant est la somme des deux précédents.
    
    \item \textbf{Syracuse (ou Suite de Collatz)} \\
    Demandez à l'utilisateur un nombre entier positif $n$. Écrivez un programme qui utilise une boucle \texttt{while} pour générer la suite de Syracuse de $n$, où chaque terme est calculé de la manière suivante :
    \[
    n = 
    \begin{cases} 
      n / 2 & \text{si } n \text{ est pair} \\
      3n + 1 & \text{si } n \text{ est impair}
    \end{cases}
    \]
    La suite continue jusqu'à ce que $n$ atteigne $1$.
    
    \item \textbf{Produit des entiers pairs} \\
    Écrire un programme qui demande un nombre entier positif $n$ et utilise une boucle \texttt{while} pour calculer le produit des entiers pairs de $2$ jusqu'à $n$. Si $n$ est impair, ignorez-le.
    
    \item \textbf{Jeu du palindrome} \\
    Écrire un programme qui demande à l'utilisateur d'entrer un mot et utilise une boucle \texttt{while} pour inverser le mot. Le programme doit ensuite indiquer si le mot est un palindrome (un mot qui se lit de la même manière dans les deux sens).
    
\end{enumerate}