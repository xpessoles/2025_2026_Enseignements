

\begin{enumerate}

    \item \textbf{Nombre parfait} \\
    Un nombre parfait est un entier positif égal à la somme de ses diviseurs positifs propres (autres que lui-même). Par exemple, $6$ est un nombre parfait car $6 = 1 + 2 + 3$. Écrivez un programme qui demande un entier positif $n$ et utilise une boucle \texttt{while} pour déterminer si $n$ est un nombre parfait.
    
    \item \textbf{Trouver les nombres premiers inférieurs à $n$} \\
    Demandez à l’utilisateur un nombre entier positif $n$. Écrivez un programme qui utilise une boucle \texttt{while} pour afficher tous les nombres premiers inférieurs à $n$. Assurez-vous d’optimiser le programme pour réduire le nombre de calculs nécessaires.
    
    \item \textbf{Approximation de $\pi$} \\
    Écrire un programme qui utilise une boucle \texttt{while} pour approximer la valeur de $\pi$ en utilisant la série de Leibniz :
    \[
    \pi = 4 \left( 1 - \frac{1}{3} + \frac{1}{5} - \frac{1}{7} + \dots \right)
    \]
    Continuez jusqu'à ce que la différence entre deux itérations successives soit inférieure à $10^{-6}$.
    
    \item \textbf{Décomposition en facteurs premiers} \\
    Écrire un programme qui demande à l'utilisateur un entier positif $n$ et utilise une boucle \texttt{while} pour décomposer $n$ en facteurs premiers. Par exemple, pour $n = 60$, le programme devrait afficher $2 \times 2 \times 3 \times 5$.
    
    \item \textbf{Nombre de Kaprekar} \\
    Un nombre de Kaprekar est un nombre $n$ tel que $n^2$ peut être décomposé en deux parties dont la somme est égale à $n$. Par exemple, $45$ est un nombre de Kaprekar car $45^2 = 2025$ et $20 + 25 = 45$. Écrire un programme qui vérifie si un nombre donné par l'utilisateur est un nombre de Kaprekar.
    
    \item \textbf{Conjecture de Syracuse} \\
    Écrire un programme qui demande à l'utilisateur un entier positif $n$ et utilise une boucle \texttt{while} pour vérifier la conjecture de Syracuse, aussi connue sous le nom de conjecture de Collatz. Comptez le nombre d'étapes nécessaires pour que $n$ atteigne $1$.
    
    \item \textbf{Trouver les nombres d’Armstrong} \\
    Un nombre d’Armstrong est un nombre qui est égal à la somme de ses chiffres chacun élevé à la puissance du nombre de chiffres. Par exemple, $153$ est un nombre d’Armstrong car $153 = 1^3 + 5^3 + 3^3$. Écrire un programme qui demande un nombre entier $n$ et affiche tous les nombres d’Armstrong inférieurs ou égaux à $n$.
    
    \item \textbf{Racine carrée par la méthode de Newton} \\
    Écrire un programme qui demande un nombre positif $x$ et utilise la méthode de Newton pour calculer la racine carrée de $x$. Continuez les itérations jusqu'à ce que la différence entre deux approximations successives soit inférieure à $10^{-6}$.
    
    \item \textbf{Ensemble de nombres parfaits inférieurs à $n$} \\
    Demandez à l’utilisateur un nombre entier positif $n$. Écrivez un programme qui utilise une boucle \texttt{while} pour trouver tous les nombres parfaits inférieurs à $n$. Les nombres parfaits sont rares, donc le programme devra être efficace.
    
    \item \textbf{Détection des cycles dans une suite} \\
    Demandez à l'utilisateur de saisir un entier $n$ et définissez la suite suivante :
    \[
    n = 
    \begin{cases} 
      n/2 & \text{si } n \text{ est pair} \\
      3n + 1 & \text{si } n \text{ est impair}
    \end{cases}
    \]
    Utilisez une boucle \texttt{while} pour vérifier si la suite entre dans un cycle. Affichez le cycle trouvé.
    
\end{enumerate}
