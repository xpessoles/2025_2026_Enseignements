

\begin{enumerate}

    \item \textbf{Trouver les éléments uniques} \\
    Écrivez un programme qui demande à l'utilisateur de saisir une liste d'éléments et affiche une nouvelle liste contenant uniquement les éléments uniques (pas de doublons).

    \item \textbf{Échange des extrémités} \\
    Écrivez un programme qui demande à l'utilisateur de saisir une liste d'éléments et échange le premier et le dernier élément de la liste. Affichez la liste modifiée.

    \item \textbf{Liste de paires} \\
    Écrivez un programme qui prend une liste d'éléments en entrée et génère une liste de paires contenant chaque élément et son indice. Par exemple, pour \texttt{['a', 'b', 'c']}, la sortie doit être \texttt{[(0, 'a'), (1, 'b'), (2, 'c')]}.

    \item \textbf{Produits d'une liste} \\
    Écrivez un programme qui demande à l'utilisateur une liste de nombres et retourne une nouvelle liste où chaque élément est le produit des autres éléments de la liste (sans l'élément courant).

    \item \textbf{Rassembler les éléments pairs et impairs} \\
    Écrivez un programme qui demande une liste de nombres à l'utilisateur et retourne deux listes : une contenant les éléments pairs et l'autre les éléments impairs.

    \item \textbf{Supprimer les valeurs nulles} \\
    Écrivez un programme qui demande à l'utilisateur de saisir une liste d'éléments (qui peut contenir des valeurs nulles) et affiche une nouvelle liste sans les valeurs nulles (\texttt{None}, \texttt{0}, \texttt{""}, etc.).

    \item \textbf{Trouver les sous-listes continues dont la somme est zéro} \\
    Écrivez un programme qui demande une liste de nombres et trouve toutes les sous-listes continues dont la somme des éléments est égale à zéro.

    \item \textbf{Trouver le sous-ensemble maximal} \\
    Écrivez un programme qui demande une liste de nombres et retourne la sous-liste avec la somme maximale.

    \item \textbf{Déplacer les zéros en fin de liste} \\
    Écrivez un programme qui demande une liste de nombres et déplace tous les zéros en fin de liste, en conservant l'ordre des autres éléments.

    \item \textbf{Liste de listes} \\
    Écrivez un programme qui demande à l'utilisateur une liste de nombres et crée une liste de listes, où chaque sous-liste contient les éléments précédents. Par exemple, pour l'entrée \texttt{[1, 2, 3]}, la sortie doit être \texttt{[[1], [1, 2], [1, 2, 3]]}.

    \item \textbf{Vérifier si une liste est palindromique} \\
    Écrivez un programme qui demande une liste d'éléments et vérifie si elle est identique à l'envers (palindrome).

    \item \textbf{Compter les éléments pairs et impairs} \\
    Écrivez un programme qui demande une liste de nombres et affiche le nombre d'éléments pairs et impairs dans la liste.

    \item \textbf{Rotation circulaire des éléments} \\
    Écrivez un programme qui demande une liste d'éléments et un nombre entier $k$. Faites une rotation circulaire de la liste de $k$ positions vers la droite.

    \item \textbf{Fusion de listes triées} \\
    Écrivez un programme qui prend en entrée deux listes triées de nombres et les fusionne en une seule liste triée.

    \item \textbf{Trouver le deuxième plus grand élément} \\
    Écrivez un programme qui demande une liste de nombres et affiche le deuxième plus grand élément de la liste.

    \item \textbf{Calcul des différences successives} \\
    Écrivez un programme qui prend une liste de nombres en entrée et crée une nouvelle liste contenant les différences entre chaque paire d'éléments consécutifs.

    \item \textbf{Diviser une liste en sous-listes de longueur égale} \\
    Écrivez un programme qui prend une liste d'éléments et un entier $n$, et divise la liste en sous-listes de longueur $n$.

    \item \textbf{Calcul de la médiane} \\
    Écrivez un programme qui demande une liste de nombres et affiche la médiane. Assurez-vous de trier la liste et gérer les listes de longueur paire et impaire.

    \item \textbf{Vérifier les doublons dans une liste} \\
    Écrivez un programme qui demande une liste d'éléments et indique si la liste contient des doublons.

    \item \textbf{Trouver les nombres manquants} \\
    Écrivez un programme qui demande une liste de nombres triée dans un intervalle donné et identifie les nombres manquants de cet intervalle.
    
\end{enumerate}

