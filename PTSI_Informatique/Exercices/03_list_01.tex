

\begin{enumerate}

    \item \textbf{Création d'une liste} \\
    Écrivez un programme qui demande à l'utilisateur de saisir 5 nombres et qui les stocke dans une liste. Affichez ensuite la liste complète.

    \item \textbf{Accéder aux éléments de la liste} \\
    Écrivez un programme qui demande à l'utilisateur de saisir une liste de 5 éléments. Affichez le premier et le dernier élément de la liste.

    \item \textbf{Trouver la longueur d'une liste} \\
    Écrivez un programme qui demande à l'utilisateur de saisir une liste d'éléments et affiche la longueur de la liste.

    \item \textbf{Additionner les éléments d'une liste} \\
    Écrivez un programme qui demande à l'utilisateur de saisir une liste de nombres et calcule la somme des éléments de la liste.

    \item \textbf{Trouver le maximum et le minimum} \\
    Écrivez un programme qui demande à l'utilisateur de saisir une liste de nombres et affiche le plus grand et le plus petit nombre de la liste.

    \item \textbf{Inverser une liste} \\
    Écrivez un programme qui demande à l'utilisateur de saisir une liste d'éléments et affiche la liste inversée.

    \item \textbf{Vérifier la présence d'un élément} \\
    Écrivez un programme qui demande à l'utilisateur de saisir une liste d'éléments et un élément à rechercher. Indiquez si l'élément est présent ou non dans la liste.

    \item \textbf{Compter les occurrences d'un élément} \\
    Écrivez un programme qui demande à l'utilisateur de saisir une liste d'éléments et un élément. Comptez le nombre de fois que cet élément apparaît dans la liste.

    \item \textbf{Ajouter un élément à la fin} \\
    Écrivez un programme qui demande à l'utilisateur de saisir une liste d'éléments et un nouvel élément. Ajoutez cet élément à la fin de la liste et affichez la liste mise à jour.

    \item \textbf{Insérer un élément à une position donnée} \\
    Écrivez un programme qui demande à l'utilisateur de saisir une liste, un élément, et une position. Insérez l'élément à la position spécifiée dans la liste et affichez la liste mise à jour.

    \item \textbf{Supprimer un élément donné} \\
    Écrivez un programme qui demande à l'utilisateur de saisir une liste d'éléments et un élément à supprimer. Retirez cet élément de la liste et affichez la liste mise à jour.

    \item \textbf{Supprimer l'élément à une position donnée} \\
    Écrivez un programme qui demande à l'utilisateur de saisir une liste et une position. Supprimez l'élément à cette position et affichez la liste mise à jour.

    \item \textbf{Trouver l'indice d'un élément} \\
    Écrivez un programme qui demande à l'utilisateur de saisir une liste d'éléments et un élément à rechercher. Affichez l'indice de la première occurrence de cet élément dans la liste.

    \item \textbf{Concaténer deux listes} \\
    Écrivez un programme qui demande à l'utilisateur de saisir deux listes d'éléments et affiche la liste résultant de la concaténation des deux listes.

    \item \textbf{Dupliquer une liste} \\
    Écrivez un programme qui demande à l'utilisateur de saisir une liste d'éléments et affiche une nouvelle liste qui est une copie de la liste initiale.

    \item \textbf{Retirer les doublons} \\
    Écrivez un programme qui demande à l'utilisateur de saisir une liste d'éléments et affiche une nouvelle liste sans doublons.

    \item \textbf{Trier une liste} \\
    Écrivez un programme qui demande à l'utilisateur de saisir une liste de nombres et affiche la liste triée dans l'ordre croissant.

    \item \textbf{Trouver les éléments communs entre deux listes} \\
    Écrivez un programme qui demande à l'utilisateur de saisir deux listes d'éléments et affiche les éléments communs aux deux listes.

    \item \textbf{Créer une liste de carrés} \\
    Écrivez un programme qui demande à l'utilisateur de saisir une liste de nombres et crée une nouvelle liste contenant les carrés de chaque nombre.

    \item \textbf{Fusionner et trier deux listes} \\
    Écrivez un programme qui demande à l'utilisateur de saisir deux listes de nombres, fusionne les deux listes, et affiche la liste fusionnée triée.

\end{enumerate}
